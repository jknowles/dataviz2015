\documentclass{tufte-handout}

% ams
\usepackage{amssymb,amsmath}

% utf8 encoding
\usepackage[utf8]{inputenc}

% graphix
\usepackage{graphicx}
\setkeys{Gin}{width=\linewidth,totalheight=\textheight,keepaspectratio}

% booktabs
\usepackage{booktabs}

% url
\usepackage{url}

% hyperref
\usepackage{hyperref}

% units.
\usepackage{units}

% pandoc syntax highlighting

% longtable

% multiplecol
\usepackage{multicol}

% lipsum
\usepackage{lipsum}

% title / author / date
\title{Data Visualization Fieldguide for Education}
\author{Jared E. Knowles}
\date{March 2015}


\begin{document}

\maketitle



\noindent \emph{Tufte's Tips}

\begin{itemize}
\itemsep1pt\parskip0pt\parsep0pt
\item
  Have a properly chosen format and design
\item
  Use words, numbers and drawing together
\item
  Reflect balance, proportion, and a sense of relevant scale
\item
  Display an accessible complexity of detail
\item
  Have a anarrative quality, tell a story about the data

  \begin{marginfigure}
   \includegraphics{talkHandout_files/figure-latex/introplot-1.pdf}
  \caption{Sometimes the story is very simple.}
  \end{marginfigure}
\item
  Draw in a professional manner, with the technical details done with
  care
\item
  Avoid content-free decoration, including \textbf{chartjunk}
\end{itemize}

\noindent \emph{Planning}

\noindent Data visualization is a tool for communicating \textbf{a
specific feature} of a dataset in an approachable and efficient
manner.\marginnote{
Data visualization is done in service of the audience. If the goal is not to convey 
information, it is art!}

\begin{itemize}
\itemsep1pt\parskip0pt\parsep0pt
\item
  Who is the audience? What is their background? Their biases?
\item
  What is the context? What would the user consider to be \emph{good
  news}, what would be \emph{bad news}? How different are these?
\item
  What does the design style say?

  \begin{marginfigure}
   \includegraphics{talkHandout_files/figure-latex/badexcel-1.pdf}
  \caption{Theme says a lot. This says: `I am in a hurry'.}
  \end{marginfigure}
\item
  Are you designing this visualization for a specific data set, or to
  display different data in a dashboard environment?
\item
  How long will this visualization last? Will it be updated next year?
  Does it need to be interactive?
\item
  What visual elements best map onto the data I have -- shapes, sizes,
  colors, or fill?
  \marginnote{How you turn dimensions in the into visual cues for your audience is everything.}
\end{itemize}

\begin{figure*}
 \includegraphics{talkHandout_files/figure-latex/diamondplot-1.pdf}
\caption{Shape and color work well for discrete data without too many levels, but one may work better than the other. Sometimes a line of best fit works best.}
\end{figure*}

\noindent \emph{Dimensionality}

\emph{If you got big data, I feel sorry for you son. you got \(99^{n}\)
problems but data viz ain't one.} \footnote{(adapted from Jay-Z, 2003)}

\begin{figure*}
 \includegraphics{talkHandout_files/figure-latex/bigplot-1.pdf}
\caption{Combine context and strategies to make comparisons easier for the user.}
\end{figure*}

\noindent \emph{Review}

\noindent \textbf{Before you submit}, check that you have:
\marginnote{Add reference lines and annotate your charts with text. Tell the viewer,
 don't expect them to reach the same conclusion as you naturally.}

\begin{itemize}
\itemsep1pt\parskip0pt\parsep0pt
\item
  \textbf{Axis labels} and a \textbf{title}: Is your purpose clear?
\item
  A \textbf{legend}: Are all symbols explained?
\item
  A \textbf{scale}: Is the relation of data points clear?
\item
  \textbf{Annotations}: Is there enough context? Are key points
  highlighted?
  \marginnote{Don't be afraid to model the data. A line of best fit or a smoother can provide a good summary to viewers to make comparisons across panels.}
\end{itemize}

\noindent Remember: If a picture is worth a thousand words, a good data
visualization must always be \emph{better than a table.}

\noindent Review this with \emph{fresh eyes}. Can someone who has not
seen this data before undertand and interpret the main story here?
\marginnote{If it is not clear, consider encoding a key feature twice -- e.g. using color and shape.}

\noindent \emph{Technical Details}

\noindent The final format and use case should inform the design:

\begin{marginfigure}
 \includegraphics{talkHandout_files/figure-latex/technologies-1.pdf}
\caption{An opinonated ordering of the tools available to do data viz. Not comprehensive or complete and your mileage will vary depending on your comfort with certain technologies.}
\end{marginfigure}

\begin{itemize}
\itemsep1pt\parskip0pt\parsep0pt
\item
  Will this be shared on the web, in a document, or in a presentation?
\item
  Will the users want to interact with the output?
\item
  Is the visualization intended to be shared and used widely, or is it
  for a short exploration?
\item
  Do others need to be able to manipulate this image?
\end{itemize}

\noindent Two main options. \textbf{Raster files} (.jpg, .png, and .gif)
have a fixed scale, aspect ratio and size -- not good for resizing or
reshaping. Easy to share.

\textbf{Vector files} (.pdf and .svg) are zoomable and adjust to
scale/aspect ratio changes. Viewable in browesrs, but not able to be
opened in PowerPoint, Word, etc.
\marginnote{If sharing matters most, go raster, if quality matters most consider vector files.}

\noindent \emph{Learn More}

\href{http://www.jaredknowles.com/presentations}{www.jaredknowles.com} •
\href{http://forums.datacope.org}{\url{http://forums.datacope.org}} •
\href{http://eagereyes.org/about}{\url{http://eagereyes.org/about}} •
\href{http://datastori.es/}{\url{http://datastori.es/}}


\end{document}
