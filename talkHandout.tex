\documentclass{tufte-handout}

% ams
\usepackage{amssymb,amsmath}

% utf8 encoding
\usepackage[utf8]{inputenc}

% graphix
\usepackage{graphicx}
\setkeys{Gin}{width=\linewidth,totalheight=\textheight,keepaspectratio}

% booktabs
\usepackage{booktabs}

% url
\usepackage{url}

% hyperref
\usepackage{hyperref}

% units.
\usepackage{units}

% pandoc syntax highlighting

% longtable

% multiplecol
\usepackage{multicol}

% lipsum
\usepackage{lipsum}

% title / author / date
\title{Data Visualization Fieldguide for Education}
\author{Jared E. Knowles}
\date{March 2015}


\begin{document}

\maketitle



Edward Tufte reminds you: \footnote{\url{http://www.edwardtufte.com/tufte/books_be}}

\begin{itemize}
\itemsep1pt\parskip0pt\parsep0pt
\item
  Have a properly chosen format and design
\item
  Use words, numbers and drawing together
\item
  Reflect balance, proportion, and a sense of relevant scale
\item
  Display an accessible complexity of detail
\item
  Have a anarrative quality, tell a story about the data

  \begin{marginfigure}
   \includegraphics{talkHandout_files/figure-latex/unnamed-chunk-1-1.pdf}
  \caption{Sometimes the story is very simple.}
  \end{marginfigure}
\item
  Draw in a professional manner, with the technical details done with
  care
\item
  Avoid content-free decoration, including \textbf{chartjunk}
\end{itemize}

\subsection{Planning}\label{planning}

Data visualization is a tool for communicating \textbf{a specific
feature} of a dataset in an approachable and efficient
manner.\marginnote{
Data visualization is done in service of the audience. If the goal is not to convey 
information, it is art!}

\begin{itemize}
\itemsep1pt\parskip0pt\parsep0pt
\item
  Who is the audience? What is their background? Their biases?
\item
  What is the context? What would the user consider to be \emph{good
  news}, what would be \emph{bad news}? How different are these?
\item
  What does the design style say?

  \begin{marginfigure}
   \includegraphics{talkHandout_files/figure-latex/unnamed-chunk-2-1.pdf}
  \caption{Theme says a lot. This says: `I am in a hurry'.}
  \end{marginfigure}
\item
  Are you designing this visualization for a specific data set, or to
  display different data in a dashboard environment?
\item
  How long will this visualization last? Will it be updated next year?
\item
  Does it need to be interactive?
\end{itemize}

\begin{figure*}
 \includegraphics{talkHandout_files/figure-latex/unnamed-chunk-3-1.pdf}
\caption{Small multiples spread the data out. But be careful, can the user draw meaningful distinctions between the groups from the data alone?}
\end{figure*}

How you turn dimensions in the into visual cues for your audience is
everything.

\subsection{Dimensionality}\label{dimensionality}

\emph{If you got big data, I feel sorry for you son. I got 99 problems
but data viz ain't one.} \footnote{(adapted from Jay-Z, 2003)}

\textbf{Small Multiples} * Allow the viewer to draw comparisons across
small plots

\textbf{Reduce the ink} * Each point has too much ``weight'', so make
them weigh less * This has another advantage of depicting uncertainty
through less depth, useful when conveying data from a statistical model
or sampling strategy

\begin{marginfigure}
 \includegraphics{talkHandout_files/figure-latex/unnamed-chunk-4-1.pdf}
\caption{Use the alpha channel - it is intuitive.}
\end{marginfigure}

\textbf{Add Reference Points} * Big data is overwhelming without
signposts, so give plenty of them to your viewer.

\begin{marginfigure}
 \includegraphics{talkHandout_files/figure-latex/unnamed-chunk-5-1.pdf}
\caption{Scale invites the user to make comparisons.}
\end{marginfigure}

\begin{itemize}
\itemsep1pt\parskip0pt\parsep0pt
\item
  Model the data
\end{itemize}

\begin{marginfigure}
 \includegraphics{talkHandout_files/figure-latex/unnamed-chunk-6-1.pdf}
\caption{But why invite, when you can make comparisons directly and show those instead? This figure shows only the loess curves of the relationship between attendance and reading scores.}
\end{marginfigure}

\section{Review}\label{review}

\textbf{Before you submit, check that you have:}

\begin{itemize}
\itemsep1pt\parskip0pt\parsep0pt
\item
  \textbf{Axis labels} and a \textbf{title}: Is your purpose clear?
\item
  A \textbf{legend}: Are all symbols explained?
\item
  A \textbf{scale}: Is the relation of data points clear?
\item
  \textbf{Annotations} Is there enough context? Are key points
  highlighted?
\end{itemize}

Review this with \emph{fresh eyes}. Can someone who has not seen this
data before undertand and interpret the main story here?

Remember: If a picture is worth a thousand words, a good data
visualization must always be \emph{better than a table.}

\subsection{Technical Details}\label{technical-details}

\textbf{The final format and use case should inform the design:}

\begin{marginfigure}
 \includegraphics{talkHandout_files/figure-latex/technologies-1.pdf}
\caption{An opinonated ordering of the tools available to do data viz. Not comprehensive or complete and your mileage will vary depending on your comfort with certain technologies.}
\end{marginfigure}

\textbf{Raster Files}

\begin{itemize}
\itemsep1pt\parskip0pt\parsep0pt
\item
  \textbf{jpg} , \textbf{png} , \textbf{gif}.
\item
  Fixed scale, aspect ratio, and size
\item
  Reasonable file size
\item
  Get blurry when you blow them up or stretch them
\item
  Viewable in almost any image viewing and editing system, including any
  modern web browser, PowerPoint, etc.
\end{itemize}

\textbf{Vector Files}

\begin{itemize}
\itemsep1pt\parskip0pt\parsep0pt
\item
  \textbf{pdf} and \textbf{svg}
\item
  Infinitely zoomable, adjustable on the fly
\item
  Larger file size
\item
  Viewable and usable in fewer systems. SVGs can be used in modern web
  browsers. PDFs included in other PDF reports.
\end{itemize}


\end{document}
